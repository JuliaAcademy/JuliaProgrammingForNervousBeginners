\documentclass[]{article}

\usepackage{amsmath}
\usepackage{amsfonts}
\usepackage{amssymb}
\usepackage{xcolor}
\usepackage{url}
\usepackage{graphicx}
\usepackage{geometry}
\usepackage[most]{tcolorbox}

% {Begin zh_TW}
\usepackage{fontspec}   % 加這個就可以設定字體
\usepackage{xeCJK}       % 讓中英文字體分開設置
\setCJKmainfont{微軟正黑體} % 設定中文為系統上的字型,而英文不去更動,使用原 TeX 字型
\XeTeXlinebreaklocale "zh"             % 這兩行一定要加,中文才能自動換行
\XeTeXlinebreakskip = 0pt plus 1pt     % 這兩行一定要加,中文才能自動換行

\defaultCJKfontfeatures{AutoFakeBold=6,AutoFakeSlant=.4} %以後不用再設定粗斜
% {End zh_TW}

\setlength{\parindent}{0em}
\setlength{\parskip}{3ex}
\geometry{left=3cm, top = 3cm, right = 3cm, bottom = 3cm}

\newcommand{\ignore}[1]{}
\newcommand{\codequote}[1]{\colorbox{lightgray}{\tt #1}}

\begin{document}

\author{}  \date{}
\title{第 1 週,第 1 課:一個最小的工作範例}
\maketitle

\vspace*{-2cm}
-----------------

\section*{目標}
\begin{description}
	\item[*] 提供一個例子以開始談論 Julia
\end{description}

-----------------

在我們的第一課中,我們廣泛借用了第 0 課---關於你必須準備好來參加本課程的內容\footnote{啟動 Julia 並進入 REPL;編輯純文字檔案並將它們儲存在本課程的專用文件夾中。}。

我還想提醒你課程口號:{\bf 小步走,沒有空隙,永遠有意義}。目的是讓緊張的初學者保持參與!

這節課我們從零開始,所以我們不能完全遵守口號的所有部分。這個想法是首先在本課中看到編碼的實際操作,然後在第 2 課中詳細解釋。

\section*{輸入 REPL 代碼}

\begin{minipage}{7cm}
打開 REPL,輸入 \colorbox{lightgray}{\tt "Hello, world"}。
\end{minipage}
\hspace{4em}\begin{minipage}{7cm}
	 示範:在本課的影片中,我們精確地向你展示如何做到這一點。
 \end{minipage}

\colorbox{lightgray}{\tt "Hello, world"} 是一個字串值。Julia 有許多其他類型的值:我們將看到其中的一些,例如數字值和字符值\footnote{即,作為字符的值。}。

\begin{minipage}{7cm}
現在輸入 \colorbox{lightgray}{\tt mystringexample1 = "Hello, world"}。
\end{minipage}
\hspace{4em} \begin{minipage}{7cm}
這稱為\emph{指派值到變數}。
\end{minipage}

重要:\colorbox{lightgray}{\tt =} 符號將右側的字串值綁定到左側的變數名稱。這在三個地方更改了你的電腦的記憶體:
\begin{description}
	\item[$\bullet$] 名稱 \colorbox{lightgray}{\tt mystringexample1} 被放入所謂的命名空間\footnote{實際上,當 Julia 運行時,它可以有多個命名空間,但這是我們在本課程中會不涉及的進階主題。一旦名稱位於命名空間中,它就會一直存在,直到你關閉整個命名空間。關閉 Julia 會話也會關閉所有命名空間。}。
  \item[$\bullet$] 字串值 \colorbox{lightgray}{\tt "Hello, world"} 被建立\footnote{ 與上面的分開但方式相同。}。
  \item[$\bullet$] 左側名稱和右側值之間的 \colorbox{lightgray}{\tt =} 符號建立了名稱和值之間的綁定。
 \item[$\bullet$] 透過將字串值綁定到名稱,Julia 將字串值儲存在你的電腦記憶體中,以便以後需要時可以使用該值。
\end{description}

\begin{minipage}{7cm}
輸入 \colorbox{lightgray}{\tt println(mystringexample1)} 。
\end{minipage}
\hspace{4em} \begin{minipage}{7cm}
	示範:\codequote{println} 是內建函數
\end{minipage}

當這一行運行時會發生什麼\footnote{人們也說:``當這一行被執行時'',以及``當這一行被評估時''。}:\\
函數 \colorbox{lightgray}{\tt println()} 接收變數名稱 \colorbox{lightgray}{\tt mystringexample1},獲取它的值(它是一個字串),重新格式化它,然後在螢幕上顯示字串並接上空行。

函數在 Julia 中非常非常重要。許多是內建的,例如 \colorbox{lightgray}{\tt println()},但 Julia 程式還建立了更多。在本課程中,你將學到很多關於 Julia 函數的知識!

\section*{建立和運行代碼檔案}

\begin{minipage}[t]{10.5cm}
	最後,建立{\tt myfirstfile.jl},作為純文字檔案(NB!),確切地包含我們上面使用的兩行代碼,將其儲存在你的課程文件夾\footnote{即建立一個 Julia 代碼檔案---此類檔案是第 0 課的主題之一。}。確保你的課程文件夾是你的工作目錄\footnote{使用 \colorbox{lightgray}{\tt pwd()} 來檢查你的工作目錄是什麼,並使用 \colorbox{lightgray}{\tt cd()} 來更改它。},然後輸入 \colorbox{lightgray}{\tt include("myfirstfile.jl")}
	\end{minipage}
\hspace{4em} \begin{minipage}[t]{4.5cm}
示範:在影片中,我們展示了結果與我們之前使用的 REPL 代碼相同。
\end{minipage}

恭喜!你的第一個 Julia 程式!編碼就是這麼簡單。

\section*{回顧與總結}

\begin{description}
	\item[*] \colorbox{lightgray}{\tt "Hello, world"} 是一個字串值。
	\item[*] \colorbox{lightgray}{\tt println()} 是一個函數。
	\item[*] \colorbox{lightgray}{\tt mystringexample1} 是一個變數名稱。
	\item[*] \colorbox{lightgray}{\tt =} 是指派:右側的值綁定到左側的名稱。
	\item[*] 函數 \colorbox{lightgray}{\tt include()} 運行它從 Julia 代碼檔案接收的代碼行。
	\item[*] Julia 代碼檔案是副檔名為 {\tt .jl} 的純文字檔案
\end{description}

我們在本課中所做的,隨著課程的進行我們將一遍遍地做:一些新想法、一些範例(你可以隨著課程進行嘗試)以及一些供你編寫和執行的代碼檔案。

請在第 1 週第 2 課之前做測驗、練習和自評作業。他們很短!在繼續之前最好先做這些。

\end{document}
