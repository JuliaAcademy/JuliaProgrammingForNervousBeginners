\documentclass[]{article}

\usepackage{amsmath}
\usepackage{amsfonts}
\usepackage{amssymb}
\usepackage{xcolor}
\usepackage{url}
\usepackage{graphicx}
\usepackage{geometry}
\usepackage[most]{tcolorbox}

% {Begin zh_TW}
% Credit: XeLaTex 中文 · GitHub
% https://gist.github.com/iamalbert/1b87a5349f7e860bfe33
\usepackage{fontspec}   % 加這個就可以設定字體
\usepackage{xeCJK}       % 讓中英文字體分開設置
\setCJKmainfont{微軟正黑體} % 設定中文為系統上的字型,而英文不去更動,使用原 TeX 字型
\XeTeXlinebreaklocale "zh"             % 這兩行一定要加,中文才能自動換行
\XeTeXlinebreakskip = 0pt plus 1pt     % 這兩行一定要加,中文才能自動換行
\defaultCJKfontfeatures{AutoFakeBold=6,AutoFakeSlant=.4} %以後不用再設定粗斜
% {End zh_TW}

\setlength{\parindent}{0em}
\setlength{\parskip}{3ex}
\geometry{left=3cm, top = 3cm, right = 3cm, bottom = 3cm}

\newcommand{\ignore}[1]{}

\begin{document}

\author{}  \date{}
\title{第 1 週,第 2 課:解構第 1 課}
\maketitle

\vspace*{-2cm}
-----------------

\section*{目標}

--- 解構第 1 課中的代碼 \\
--- 提供有關 Julia 中表示式的細節 \\
--- 傳達有效代碼的重要性 \\
--- 開始仔細學習 Julia 語言的過程

學完本課,您將能夠 \vspace{-3ex} \begin{description}
	\item[*] 舉幾個 Julia 中有效表示式的例子
	\item[*] 根據名稱和值解釋變數的結構
	\item[*] 準確解釋 Julia 中的字串值是什麼
	\item[*] 廣義地解釋什麼是函數,以及 Julia 如何識別函數
  \item[*] 解釋運算子與其他類型函數的區別
	\item[*] 舉例說明一些 Julia 分隔符號及其用法
\end{description}

\section*{複習:第1課的代碼}

在第 1 課中,您在 REPL 中運行了以下代碼

\colorbox{lightgray}{\tt mystringexample1 = "Hello, world"}  \hspace{3cm} \ldots\ 行 1\\
\colorbox{lightgray}{\tt println(mystringexample1) } \hspace{4.3cm} \ldots\ 行 2

您還建立了一個檔案,我們將其稱為 {\tt myfile.jl},儘管您可能使用了不同的名稱\footnote{在本課中,每當我們提到 {\tt myfile.jl},請注意我們指的是您建立的文件,並在您的腦海中(以及在代碼範例中)將其替換為您自己的檔案名。}。

\colorbox{lightgray}{\tt include("myfile.jl")} \hspace{5.5cm} \ldots\ 行 3

\section*{解構第 1 行,它建立了一個變數}



如何閱讀第 1 行:它有一個左手邊,一個右手邊,用等號連接它們。

右側是一個字串值(更多內容見下文):\colorbox{lightgray}{\tt "Hello, world"}。

左側是變數的名稱:\colorbox{lightgray}{\tt mystringexample1}。

\colorbox{lightgray}{\tt =} 符號將變數名稱綁定到給定的值。

這實際上改變了電腦中的一些記憶體。就技術上來說,等號是一種特殊的函數,即運算子,它的全稱是``指派運算子''\footnote{所有電腦語言都至少有一個指派運算子。它們在影響電腦記憶體的細節方面有所不同,但在本課程中,我們先遠離這些細節。你需要知道的是,在 Julia 中,指派運算子將其右側的值綁定到其左側的變數名稱。}。

我們說第 1 行\emph{建立}了變數,因為在第 1 行之前 \colorbox{lightgray}{\tt mystringexample1} 不是命名空間的一部分。我們可以指派一個新值,例如使用此行

\colorbox{lightgray}{\tt mystringexample1 = "a new value"}

在這種情況下,我們不會建立變數,只是將現有名稱綁定到新值。這個新值不必是字串,順便說一下,它可以是一個數字

\colorbox{lightgray}{\tt mystringexample1 = 1.1111}

或者實際上是 Julia 可以使用的任何其他類型的值。如第 1 課所述,名稱不能從命名空間中刪除,只能新增。在本課程中,我們僅使用最上層命名空間,該命名空間的存在與你的 Julia 會話一樣長。

\section*{解構字串值}

字串是一個字元序列。如你所見,在 Julia 中,我們透過將字串值括在一對雙引號中來表示它\footnote{Julia 有很多種類的值;在本課程中,我們只討論其中的幾個。請參閱第\ldots 課中對類型的討論。}。

我們之前已經簡要討論過字元。請注意,在 Julia 中,字元總是用單引號括起來。

[展示: \colorbox{lightgray}{\tt a = 'a', b = "a"}]

在本課程中,我們將僅使用國際鍵盤的一次按鍵\footnote{可能透過 Ctrl 或 Tab 鍵修改}可用的字元來形成字串。但是更多的字元在 Julia 中是有效的\footnote{包括希臘語、阿拉伯語、梵語等字母表。事實上,在 Julia 中,不僅字母字元而且非字母字元(例如用於書寫普通話的字元)都是有效字元。} 稍後我們將簡要向你展示如何輸入它們。你可能希望開始在變數和函數名稱中使用它們,但學習這樣做是你自己的專案,它不是本課程的一部分。

\section*{解構變數名稱}

只有部分在 Julia 字串值中可用的字元被允許作為變數名稱。

變數名稱必須以字母開頭\footnote{以及其他一些字元,但尤其不是數字,請參閱 Julia 文件了解詳細資訊。} 並且必須以字母、數字或底線或驚嘆號接續。在本課程中,我們僅使用羅馬字母表中的字母,但 Julia 接受的字母多於這些字母。

最佳做法是僅使用小寫字母和數字,並使用描述性名稱,例如 \colorbox{lightgray}{\tt mystringexample1}\footnote{Julia 社群通常遵守此規則。這使得驚嘆號具有非常特殊的含義,我們稍後會看到。儘管規則允許,但強烈不鼓勵在用戶代碼中使用底線;這個想法是將底線限制為 Julia 內部的特殊含義。}。

\section*{第 1 行是一個有效的 Julia 表示式}

這一點非常重要:當 Julia 處理有效代碼時,電腦會發生變化---其中一些變化達到了代碼的目的(列印輸出、計算、圖片 \ldots)。第 1 行是有效的,因為:

\begin{description}
	\item[$\bullet$] 第 1 行包含 Julia 能識別的符號。
	\item[$\bullet$] 這些符號組合成三個有效群組。
	\item[$\bullet$] 這三個有效群組組合成一個有效表示式。
\end{description}

問題:Julia 能識別哪些符號? \\
答:非常多! 但在本課程中,我們將只使用標準國際鍵盤上可見的符號,所有這些符號 Julia 都能識別。

問題:第 1 行中哪些是有效的符號群組?\\
答:三個,分別是 \colorbox{lightgray}{\tt mystringexample1}、\colorbox{lightgray}{\tt =} 和 \colorbox{lightgray}{\tt "Hello, world"}。也就是說,Julia 識別出一個有效的變數名稱、一個有效的運算子和一個有效的字串\footnote{之間的空格不是必需的,正如 Julia 從其他線索中識別的那樣。然而,空格的存在/不存在有時在 Julia 中確實很重要,我們稍後會看到。}。

問題:為什麼這個有效符號群組的組合是一個有效的表示式?\\
答:因為 \colorbox{lightgray}{\tt =} 運算子可以這樣運作。實際上,它\emph{只}能這樣運作:左邊的名字,中間的 \colorbox{lightgray}{\tt =} 和右邊的值。

只有當表示式的所有部分都被 Julia 正確識別並按照 Julia 的規則組合時,我們才有一個有效的 Julia 代碼表示式。生成有效代碼的規則非常嚴格,我們將在第 3 課中討論為什麼會這樣。

最後,讓我們注意第 1 行的某些部分本身就是有效代碼,即名稱和值。一個有效的表示式可以是一些更大的有效表示式的一部分。

\subsection*{評估無效代碼會產生錯誤訊息}
這裡有一些行話:我們說 Julia 評估它得到的每個表示式。這僅僅意味著 Julia 會嘗試執行代碼指示它執行的操作。

例如以下幾行無效代碼

\begin{minipage}{7cm}
	\colorbox{lightgray}{\tt =}\\
	\colorbox{lightgray}{\tt = "Hello, world"}\\
	\colorbox{lightgray}{\tt mystringexample1 =}\\
	\colorbox{lightgray}{\tt mystringexample1 "Hello, world"}\\
	\colorbox{lightgray}{\tt "Hello, world" = mystringexample1}
\end{minipage}

從 Julia 產生錯誤訊息,僅此而已\footnote{這不能完全保證,但絕對是 Julia 的創造者想要的!} 。在第 3 課中,我們將開始閱讀錯誤訊息和除錯無效代碼。

\subsection*{一個微妙的解釋}

然而,這可能會讓你大吃一驚,

\colorbox{lightgray}{\tt Hello, = mystringexample1}

是有效代碼。這不是因為 \colorbox{lightgray}{\tt Hello,} 是一個有效的變數名稱,而是因為逗號的特殊作用。在這裡,因為它遵循指派運算子左側的有效名稱,所以逗號表示 Julia 應該執行多重指派。讓我們用一個左邊有兩個名字的例子:

\colorbox{lightgray}{\tt Hello, world = mystringexample1 } \hspace{4.5cm} \ldots\ 行 4

\begin{minipage}[t]{8cm}
請注意,第 4 行引入了一種新的指派形式:右邊不是一個值,而是一個變數名稱。沒問題,Julia 只是使用綁定到變數名稱的值。
\end{minipage}
\hfill
\begin{minipage}[t]{6cm}
展示:我們評估第 4 行,詢問它建立的變數名稱和值
\end{minipage}

多重指派的運作方式如下:在 \colorbox{lightgray}{\tt =} 的左側,變數用逗號分隔\footnote{如你所見,你可以只有一個變數後跟逗號,然後是 = 符號。}。獲得列表後,Julia 會在右側查找值。單個字串值可以提供多個單獨的值可能看起來很奇怪,但請記住,字串是一個字元序列。由於左側是一個序列,Julia 將右側視為一個序列。如你所見,右側的多餘項被忽略。

以這種方式使用逗號進行多重指派是 Julia 允許你建立非常緊湊且通常也很容易閱讀的代碼的方式之一。如果做得好,它確實可以幫助你編寫其他人喜歡閱讀的代碼,這可以極大地簡化協作。這包括與自己合作,幾個月或幾年後,當你嘗試使舊代碼適應新用途時。

\hspace*{1.5cm}\fbox{ \begin{minipage}{12 cm}
	不用擔心在 Julia 中編寫有效表示式的難度。正如你在第 1 課中看到的,這很容易。與任何電腦語言一樣,Julia 允許極長的有效表示式。是的,要確保這樣的表示式有效可能非常困難。但這一切都無關緊要。你可以用簡短的表示式做大量的事情!

	而且你不必一次學習所有的 Julia。在本課程中,我們將逐步向你介紹越來越多的有效表示式形成方法,任何時候都不會太多。
	\end{minipage}
}

\section*{第 2 行呼叫一個函數}

當 Julia 代碼告訴一個函數做某事時,我們說它呼叫了這個函數。這裡我們使用一些輸入呼叫函數 \colorbox{lightgray}{\tt println}。這輸入是 \colorbox{lightgray}{\tt mystringexample1},
即變數的名稱。

要使此代碼成為有效代碼,\colorbox{lightgray}{\tt println} 必須能夠格式化該變數的值\footnote{像\colorbox{lightgray}{\tt "Hello, world"} 這樣的普通字串是所有值中最容易格式化的。}。在 Julia 中呼叫函數會使事情發生。

你應該思考一下:函數呼叫\colorbox{lightgray}{\tt println(mystringexample1)}做了幾件事:首先它接受變數名稱,然後獲取變數的值,然後格式化它---在這種情況下,幾乎沒有什麼可做的---然後它會在螢幕上列印格式化的字串,然後在下一個\colorbox{lightgray}{\tt julia>}提示符之前顯示一個空行。

Julia 是如何知道一段代碼引用了一個函數的?很簡單:代碼是一個有效的變數名稱,後面緊跟括號 --- 也就是說,函數名稱後面沒有任何空格,下一個字元是\colorbox{lightgray}{\tt(}。之後是函數的輸入,輸入之後是關閉的\colorbox{lightgray}{\tt)}。

然而,\colorbox{lightgray}{\tt println} 實際上並不需要任何輸入:[展示:\colorbox{lightgray}{\tt println()}, \colorbox{lightgray}{\tt include()}]。你會看到 \colorbox{lightgray}{\tt println()} 的行為就好像你給了它一個空字串:它列印一行空的內容,然後跳到新行,然後跳到 \colorbox{lightgray}{\tt julia>} 提示符 . 另一方面,\colorbox{lightgray}{\tt include()} 會引發錯誤。

規則有一個例外,即 Julia 透過名稱後面的 \colorbox{lightgray}{\tt (} 識別函數。正如我們所指出的,運算子是一種特殊的函數。它們(大部分)是數學符號,例如 \colorbox{lightgray}{\tt - }、\colorbox{lightgray}{\tt +}、\colorbox{lightgray}{\tt >} 以及它們形成的表示式具有非常接近通常數學含義的規則\footnote{但請注意:數學含義只是一個指示。必須嚴格遵守規則,所以你必須知道它們!}。在本課程中,你將學習更多的 Julia 內建函數,並且你還將編寫自己的。很大一部分 Julia 代碼是透過函數編寫的。

\section*{最終解構:分隔符}

我們在 \colorbox{lightgray}{\tt println()} 中使用的括號在 Julia 中起著重要作用:它們是分隔符。它們用於告訴 Julia 函數的輸入在哪裡開始和結束。同樣,字串周圍的雙引號是分隔符,逗號用於進行多重指派時也是如此。

請注意你輸入的內容以產生有效代碼:使用有效字元,你輸入了值、名稱、運算子和分隔符。這涵蓋了我們在本課程中使用的所有有效代碼\footnote{如前所述,我們僅使用 Julia 中有效字元的一小部分。同樣,我們只使用了 Julia 的一些運算子和分隔符。}。

%\newpage
\section*{回顧與總結}

\begin{description}
	\item[$\bullet$] Julia 代碼由有效的表示式組成。
	\item[$\bullet$] 在第 1 週,我們使用包含值、名稱、運算子和分隔符的有效表示式。
	\item[$\bullet$] Julia 中的名稱必須以字母開頭,並以字母或數字或底線或驚嘆號接續。
	\item[$\bullet$] 變數是綁定到值的名稱。
	\item[$\bullet$] 呼叫函數意味著在其名稱後面加上括號,將傳遞給函數的值和/或變數名稱括起來。
	\item[$\bullet$] 運算子是一種不需要括號的特殊函數---通常它們是像 \colorbox{lightgray}{\tt =}、 \colorbox{lightgray}{\tt +}、 \colorbox{lightgray}{\tt <} 等符號。
	\item[$\bullet$] 字串值周圍的 \colorbox{lightgray}{\tt " "} 對,和函數輸入周圍的括號 \colorbox{lightgray}{\tt ( )} 等分隔符有助於構建 Julia 代碼。
\end{description}

\end{document}
